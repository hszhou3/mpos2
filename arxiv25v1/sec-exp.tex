
\section{Implementation and Evaluation}
\label{sec:impl-eval}

This section\footnote{See {\em Experimental results} from the ``New Cryptarchia'' at \url{https://www.notion.so/nomos-tech/New-Cryptarchia-202261aa09df8026b31ad5e09c1a3fbb}. The companion Jupyter notebook is available at \url{https://github.com/logos-co/nomos-pocs/blob/cryptarchia-v2/cryptarchia-v2/cryptarchia-v2.ipynb}.} describes our prototype implementation and the empirical evaluation of \ProjBase. 

%\We follow the \emph{Experimental results} from the ``New Cryptarchia (July Version)'' design notes (MD/PDF) and its 

\subsection{Implementation Overview}
We implemented a discrete-event simulator that models validators, network propagation, and the \ProjBase fork-choice. Each block stores $(\id,\val,\slot,\txs,\refs,y,\pi,\sigma)$ as in Alg.~\ref{alg:block-creation}. References are maintained as adjacency lists of the ref-DAG; dependencies are checked against a UTXO set. The simulator exposes:
\begin{itemize}
  \item \textbf{Eligibility:} VRF-based slot eligibility $\Eligibility(v,s)$ (Sec.~\ref{subsec:notation}); multiple proposers per slot emerge from the Bernoulli trials across validators.
  \item \textbf{Reference selection:} Within the window $w$, each producer attempts to select a large antichain of parents. We provide two backends: (i) a greedy antichain builder; and (ii) an (optional) exact Dilworth-based maximum-antichain\footnote{For the exact antichain computation via Dilworth's theorem, please see Appendix~\ref{sec:appendix-dilworth}.} routine on the transitive reduction when the window is small.
  \item \textbf{Long-ref:} At most one \emph{long-ref} $\ell$ to a block outside the window (if any unreachable component is detected). Long-refs carry \emph{zero} weight in fork-choice (Sec.~\ref{sec:cryptareon-base}).
  \item \textbf{Conflict resolution:} CCA-based branch selection using the window-filtered reference count $\wref(\cdot;w)$ (Alg.~\ref{alg:cca-resolve}). In our default runs $\contrib(d)\equiv 1$; optional stake-weighting is disabled unless stated.
  \item \textbf{Adversary:} An adaptive, withholding adversary that (i) hides its blocks; (ii) at each honest conflict computes an \emph{ILP-optimized} legal DAG from the CCA to maximize branch weight; and (iii) releases its DAG to attempt a reorg.
\end{itemize}

\subsection{Experimental Setup}
Unless stated otherwise, the default parameters are:
\begin{center}
\begin{tabular}{lcl}
\toprule
Broadcast delay mean & = & $0.5$ s\\
Dissemination delay mean & = & $0.5$ s\\
Blend hops & = & $3$\\
Proof-of-leadership time & = & $1$ s\\
Reference window $w$ & = & $30$ slots\\
Production amplification $f$ & = & $0.25$\\
Adversary stake/control & = & $0.30$ \\
Validators & = & $1000$ \\
\bottomrule
\end{tabular}
\end{center}
The network model combines “Blend”-style neighbor blending with per-hop dissemination; honest nodes use best-effort antichains given current visibility; transaction dependencies (UTXO conflicts) are enforced.

\subsection{Metrics}
We track: (i) \textbf{reorg length} (depth of the deepest reverted block); (ii) \textbf{time to stabilization} (slots until last reorg across a horizon); (iii) \textbf{finality proxy} (age after which reorg probability $<10^{-k}$ for a fixed $k$); (iv) \textbf{short-ref density} (average $\wref$ per block); (v) \textbf{throughput proxy} (blocks per wall-clock time under the $f$ setting); and (vi) \textbf{concurrent tip count} (size of $\Tips(G)$ over time).

\subsection{Results Summary}
Across the benchmarks, our findings are:
\begin{itemize}
  \item \textbf{Stability under $30\%$ adversary.} The network consistently resists optimized reorg attempts; binned histograms of reorg lengths are time-equivalent to Cryptarchia blocks.
  
  \item \textbf{Effect of production rate $f$.} At $f{=}0.5$ (${\approx}10\times$ Cryptarchia), short-term instability increases but reorgs resolve faster; at $f{=}0.15$ (${\approx}3\times$ Cryptarchia), behavior is more stable---supporting the intuition that higher information rate yields quicker convergence at the cost of local variance.
  \item \textbf{Stronger adversaries.} With $40$--$45\%$ adversarial control, \ProjBase still shows bounded reorgs that compare favorably to Cryptarchia at equal stake and network settings.
  \item \textbf{Near-majority and majority.} At $49\%$, the system degrades gracefully; at ${\ge}51\%$ the attacker predictably dominates (as expected for a majority adversary), though the induced reorg distributions remain informative.
\end{itemize}

\iffalse
\subsection{Empirical Evidence for Tip Boundedness}
To support the analytic TB lemmas, we instrument the simulator to sample $|\Tips(G_t)|$ each slot $t$ and compute distributional statistics over long runs (e.g., $10^6$ slots). We report:
\begin{itemize}
  \item the time series of concurrent tip count and its moving average;
  \item the empirical cumulative distribution function (CDF) of $|\Tips(G_t)|$; 
  \item the $95$th/ $99$th percentile of $|\Tips(G_t)|$ as a function of $(\Delta, w, f)$;
  \item cross-plots of concurrent tips versus reorg length to show correlation.
\end{itemize}
In all configurations where $w\ge \Delta$ and honest stake $> 1/2$, the empirical $95$th percentile of concurrent tips remains bounded by a small constant times $\Delta \cdot$ (honest block rate), consistent with the predicted $\beta=O(\Delta \cdot \text{block rate})$. When $w<\Delta$, the bound degrades and outliers appear, indicating that window size must cover propagation delay to sustain TB with high probability. These measurements reinforce that TB is not only theoretically necessary but also \emph{observably} maintained under the recommended parameter regimes.
\fi

\subsection{Reorg Distributions vs.\ Stake and Latency}
We next assess how the attacker’s capability (stake fraction and network control) affects the distribution of reorg depths and times. We vary $\alpha$ from $0.1$ to $0.49$ and test $\Delta$ values from $0.2$ s to $2$ s (holding other parameters fixed). For each setting, we simulate long runs (up to $10^6$ slots) to gather a distribution of reorg lengths and finalization times.

\paragraph{Impact of Adversarial Stake.}
As adversarial stake $\alpha$ increases toward $0.5$, the frequency and length of reorgs naturally increase. However, even at $\alpha=0.45$, we observe that reorg lengths remain bounded (e.g., $\le 10$ slots deep with $>99\%$ probability) under $w=30$. The tail distribution of reorg depth grows sharply as $\alpha\to 0.5$, consistent with theoretical loss of safety at majority. For example, at $\alpha=0.49$, occasional reorgs of depth $15$--$20$ slots occur (though rare). Still, for $\alpha < 0.5$ the system shows a clear separation between typical reorgs (depth $\le 3$) and worst-case extremes (controlled by the $\Delta$ and $w$ relationship).

\paragraph{Impact of Network Delay.}
Holding $\alpha$ constant (say $0.3$), we increase network latency $\Delta$ relative to slot time. As long as $w \ge \Delta$, the reorg depth distribution shifts only slightly: higher $\Delta$ causes more short forks (transient tips) but those forks are resolved within $w$ slots. If $w < \Delta$, however, reorgs can grow arbitrarily: indeed, for $w=20$ slots and $\Delta=30$ slots of delay, we see unbounded tip oscillation (safety fails). This empirically confirms the necessity of $w\ge \Delta$ for security, matching our theoretical TB requirement.

\paragraph{Finality Time vs.\ Stake/Delay.}
Using the finality proxy (age after which probability of reorg $<10^{-k}$), we find that finality time remains on the order of tens of slots across the range of $\alpha$ tested, as long as $\alpha<0.5$ and $w\ge \Delta$. For example, at $\alpha=0.3$, $\Delta=0.5$ s, $w=30$, we achieve $k=6$ (six nines of confirmation) within roughly $30$--$40$ slots. At $\alpha=0.45$, this increases to about $60$ slots for the same confidence level, reflecting slower convergence under heavier adversarial presence. Nonetheless, the system consistently finalizes in $O(w)$ slots in all secure regimes.


%%%%%%%%%%%%%%%%%%%%%%%%%%%%%%%%%%%%%%%%%%%%%%%%%%%%%%%%%%%%%%%%%%%%%%%%%%%%%%%
% Section 6: Related Work
%%%%%%%%%%%%%%%%%%%%%%%%%%%%%%%%%%%%%%%%%%%%%%%%%%%%%%%%%%%%%%%%%%%%%%%%%%%%%%%
\section{Related Work}
\label{sec:related}

\iffalse
Consensus research has evolved along two main lines: chain-based protocols, which analyze linear blockchains via backbone properties such as common prefix, chain growth, and chain quality, and DAG-based protocols, which leverage parallel block production but often lack equally clear invariants. Cryptareon builds on both traditions. From chain-based work we adopt the backbone methodology, but we extend it with DAG-specific invariants (DG, DQ, DCP, TB) to capture concurrency. From DAG-based systems we inherit the advantages of multi-proposer parallelism, but we differ by grounding our design in formal invariants that connect directly to persistence and liveness. This section situates Cryptareon within these two streams of work, highlighting both the conceptual lineage and the novel contributions.
\fi

Decentralized consensus has evolved from single-leader PoW/PoS chains to leaderless and DAG-based designs. We briefly review key areas:

\subsection{Proof-of-Stake Chain Protocols}
Early PoS protocols (e.g., Snow White~\cite{FC:DaiPasShi19}, Ouroboros Praos~\cite{EC:DGKR18}, Genesis~\cite{CCS:BGKRZ18}) follow a longest-chain model with one leader per slot (chosen via VRFs). These guarantee security under limited throughput. BFT-style PoS systems like Algorand~\cite{Algorand}, HotStuff~\cite{HotStuff}, and Dfinity~\cite{Dfinity} use committees to vote on blocks, but often incur higher communication costs. Hybrid protocols with VDFs~\cite{FC:DebCabTse21} improve unpredictability at the cost of extra delay. Privacy-preserving PoS (Ouroboros Crypsinous~\cite{SP:KKKZ19}) adds anonymity, but is mostly orthogonal to consensus structure. A common challenge is unpredictability: if leaders or committees can be precomputed, adversaries may launch selfish-mining or bribery attacks~\cite{FKLTZ24,rationalattacksPOS}. Our work orthogonalizes these concerns by treating eligibility as a black box (modeled by VRFs or ideal coins).

\subsection{Multi-Proposer and DAG Protocols}
Protocols like PHANTOM/GHOSTDAG~\cite{AFT:SWZ21} and Conflux extend Nakamoto consensus to DAGs by sorting or forking rules. PHANTOM solves a global optimization, while GHOSTDAG greedily orders the DAG. Prism~\cite{Prism} uses multiple chains and a robust ordering layer. Spectre~\cite{EPRINT:MorKulYok18} and Tree-Graph~\cite{EPRINT:ZhaChaLeo18} also weight DAGs. Avalanche~\cite{SnowFamily} achieves probabilistic consensus via repeated subsampled voting on a DAG. Lachesis (Fantom) uses PoS-weighted asynchronous BFT on a DAG to achieve instant finality. These systems show DAGs enable high throughput and finality, but their analyses and assumptions differ. Our \Proj builds on this lineage by using a CCA-weighted fork choice and sliding window, combining ideas of GHOST and GHOSTDAG with stake weighting and finality anchors. %We also consider transaction validity and slashing within the DAG, aspects often omitted in earlier DAG designs.

\subsection{Consensus Analysis}
The Bitcoin Backbone framework~\cite{EC:GarKiaLeo15} established CP/CG/CQ for PoW chains; similar analysis applies to PoS chains~\cite{EC:PasSeeShe17}. DAG protocols challenge these proofs because blocks do not lie on a single chain. Recent work (e.g., Fantom's Lachesis analysis, Narwhal-Tusk~\cite{NarwhalTusk}) decouple availability from ordering. Our security proof adapts backbone-style coupling arguments to the DAG setting. We introduce DAG invariants (DG/DQ/DCP/TB) and show that after GST, honest blocks cross-link each other within the window, ensuring one branch's weight monotonically grows. These arguments align with known results (e.g., \cite{EC:GarKiaLeo15} for chain CP, \cite{AFT:SWZ21} for DAG connectivity), but require new steps to handle parallel blocks and CCA choices.

\subsection{Economic and Practical Perspectives}
Multi-proposer designs also impact incentive structure: they can improve short-term censorship resistance and fairness~\cite{FOCIL, multiFee}. Our eligibility and confirmation scheme is compatible with various fee distribution policies. Additionally, implementing VRF eligibility and antichain selection must be done efficiently; \ProjBase's algorithms are designed for practicality (e.g., using greedy references and PRF sampling of transactions).


In summary, while chain-based protocols provide clear analytic frameworks and DAG-based
protocols offer scalability through parallelism, existing work has not fully reconciled the two.
Cryptareon advances this reconciliation by grounding a leaderless DAG design in backbone-style
invariants that yield persistence and liveness. Building on this foundation, the next section
explores future directions, including refined invariant formulations, stronger adversarial models,
and practical deployment considerations.


\paragraph{Future Directions.} We have presented \ProjIdeal and \ProjBase; extending to \ProjFull (with tighter window management and optional long-reference policies) is ongoing work. The ultimate goal is a fully formalized, deployed multi-proposer consensus that rivals chain-based systems in simplicity, while offering superior performance.




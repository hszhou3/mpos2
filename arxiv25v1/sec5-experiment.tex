


\section{Implementation and Evaluation}
\label{sec:impl-eval}

This section describes our prototype implementation and the empirical evaluation of \ProjBase. For background narratives and figures, see the documents titled “New Cryptarchia” and “Simulations New Cryptarchia.” The companion Jupyter notebook is referenced in those documents.

\subsection{Implementation Overview}
We implemented a discrete-event simulator that models validators, network propagation, and the \ProjBase{} fork-choice. Each block stores
$(\id,\val,\slot,\txs,\refs,y,\pi,\sigma)$ as in Alg.~\ref{alg:block-creation}. References are maintained as adjacency lists of the ref-DAG; dependencies are checked against a UTXO set.

The simulator exposes:
\begin{itemize}
	\item \textbf{Eligibility:} VRF-based slot eligibility $\Eligibility(v,s)$ (Sec.~\ref{subsec:notation}); multiple proposers per slot emerge from Bernoulli trials across validators.
	\item \textbf{Reference selection:} Within the window $w$, each producer attempts to select a large antichain of parents. We provide two backends: (i) a greedy antichain builder; and (ii) an (optional) exact Dilworth-based maximum-antichain routine on the transitive reduction when the window is small (see Appendix~\ref{sec:appendix-dilworth}).
	\item \textbf{Long-ref:} At most one \emph{long-ref} $\ell$ to a block outside the window (if any unreachable component is detected). Long-refs carry \emph{zero} weight in fork-choice (Sec.~\ref{sec:base}).
	\item \textbf{Conflict resolution:} CCA-based branch selection using the window-filtered reference count $\wref(\cdot;w)$ (Alg.~\ref{alg:cca-resolve}). In our default runs $\contrib(d)\equiv 1$; optional stake-weighting is disabled unless stated.
	\item \textbf{Adversary:} An adaptive, withholding adversary that (i) hides its blocks; (ii) concentrates all adversarial stake in a single node-identity; (iii) observes the honest network with effectively zero delay; (iv) at each honest conflict computes either an \emph{ILP-optimized} legal DAG (optimal sliding-window, exhaustive local) or a \emph{global heuristic} DAG; and (v) releases its DAG to attempt a reorg.
\end{itemize}

\subsection{Experimental Setup}
Unless stated otherwise, the default parameters are:
\begin{center}
\begin{tabular}{lcl}
\toprule
Broadcast delay mean & = & $0.5$ s\\
Dissemination delay mean & = & $0.5$ s\\
Blend hops & = & $3$\\
Proof-of-leadership time & = & $1$ s\\
Reference window $w$ & = & $30$ slots\\
Production amplification $f$ & = & $0.25$\\
Adversary stake/control & = & $0.30$ \\
Validators & = & $1000$ \\
\bottomrule
\end{tabular}
\end{center}

The network model combines “Blend”-style neighbor blending with per-hop dissemination; honest nodes use best-effort antichains given current visibility; transaction dependencies (UTXO conflicts) are enforced. In worst-case runs, the adversary is modeled as a single identity with perfect intra-adversary coordination and zero observation delay to the honest DAG; this constructs a conservative baseline.

\subsection{Metrics}
We track: (i) \textbf{reorg length} (depth of the deepest reverted block); (ii) \textbf{time to stabilization} (slots until last reorg across a horizon); (iii) \textbf{finality proxy} (age after which reorg probability $<10^{-k}$ for a fixed $k$); (iv) \textbf{short-ref density} (average $\wref$ per block); (v) \textbf{throughput proxy} (blocks per wall-clock time under the $f$ setting). %; and (vi) \textbf{concurrent tip count} (size of $\Tips(G)$ over time).

\subsection{Results Summary}
Across the benchmarks, our findings are:
\begin{itemize}
	\item \textbf{Stability under $30\%$ adversary.} The network consistently resists optimized reorg attempts; binned histograms of reorg lengths are time-equivalent to Cryptarchia v1 blocks.
	\item \textbf{Quantitative comparison vs.\ Cryptarchia v1 (C1).} Under the most effective adversarial optimization tested (ILP-based, optimal sliding-window, exhaustive local), C1 produces roughly 15-block reorgs with frequency around $10^{-2}$, whereas \ProjBase{} produces 14--15-block reorgs with frequency around $1.5\times 10^{-5}$ (about $600\times$ improvement). For 10-block reorgs, C1 frequency is about $7\times 10^{-1}$ versus \ProjBase{} about $8\times 10^{-5}$ (about $8750\times$ improvement).
	\item \textbf{ILP vs.\ global heuristic.} ILP-based optimization (exhaustive local within the window) yields longer rare reorgs but is computationally expensive; the global heuristic scales to many more attacks but exhibits shorter reorgs on average, aligning with its lesser optimization power.
	\item \textbf{Effect of production rate $f$ (parallelism).} Increasing $f$ raises short-term instability but \emph{decreases} reorg length, leading to faster resolution: at $f{=}0.5$ (about $10\times$ C1), short forks are more frequent yet resolve quickly; at $f{=}0.15$ (about $3\times$ C1), behavior is more stable. Overall, higher information rate increases local variance but accelerates convergence.
	\item \textbf{Stronger adversaries.} With $40$--$45\%$ adversarial control, \ProjBase{} maintains bounded reorgs and compares favorably to C1 under the same conditions.
	\item \textbf{Near-majority and majority.} At $49\%$, the system degrades gracefully; at $\ge 51\%$ the attacker predictably dominates (as expected for a majority adversary), yet the induced distributions remain informative for parameterization and risk analysis.
\end{itemize}

\subsection{Parallelism vs.\ Reorg Length}
Empirically we observe an inverse correlation between parallelism (higher $f$) and reorg length, especially visible in large-scale heuristic runs that permit substantially more attack attempts. While ILP-based local optima can surface rare long reorgs in small samples, the broader trend across $f\in\{0.25,0.3,0.35,0.4,0.5\}$ shows decreasing tail depth as $f$ increases, consistent with the “more votes, faster convergence” intuition underpinning the design.

\subsection{Reorg Distributions vs.\ Stake and Latency}
We next assess how the attacker’s capability (stake fraction and network control) affects the distribution of reorg depths and times. We vary $\alpha$ from $0.1$ to $0.49$ and test $\Delta$ values from $0.2$ s to $2$ s (holding other parameters fixed). For each setting, we simulate long runs (up to $10^6$ slots) to gather a distribution of reorg lengths and finalization times.

\paragraph{Impact of Adversarial Stake.}
As adversarial stake $\alpha$ increases toward $0.5$, the frequency and length of reorgs naturally increase. However, even at $\alpha=0.45$, we observe that reorg lengths remain bounded (e.g., $\le 10$ slots deep with $>99\%$ probability) under $w=30$. The tail distribution of reorg depth grows sharply as $\alpha\to 0.5$, consistent with theoretical loss of safety at majority. For example, at $\alpha=0.49$, occasional reorgs of depth $15$--$20$ slots occur (though rare). Still, for $\alpha < 0.5$ the system shows a clear separation between typical reorgs (depth $\le 3$) and worst-case extremes (controlled by the $\Delta$ and $w$ relationship).

\paragraph{Impact of Network Delay.}
Holding $\alpha$ constant (say $0.3$), we increase network latency $\Delta$ relative to slot time. As long as $w \ge \Delta$, the reorg depth distribution shifts only slightly: higher $\Delta$ causes more short forks (transient tips) but those forks are resolved within $w$ slots. If $w < \Delta$, however, reorgs can grow arbitrarily: for example, with $w=20$ slots and $\Delta=30$ slots of delay, tip oscillation becomes effectively unbounded (safety fails). This empirically confirms the necessity of $w \ge \Delta$ for security, matching the TB requirement.

\paragraph{Finality Time vs.\ Stake/Delay.}
Using the finality proxy (age after which probability of reorg $<10^{-k}$), finality time remains on the order of tens of slots across the range of $\alpha$ tested, as long as $\alpha < 0.5$ and $w \ge \Delta$. For example, at $\alpha=0.3$, $\Delta=0.5$ s, $w=30$, we achieve $k=6$ within roughly $30$--$40$ slots. At $\alpha=0.45$, this increases to about $60$ slots for the same confidence level, reflecting slower convergence under heavier adversarial presence. The system consistently finalizes in $O(w)$ slots in secure regimes.


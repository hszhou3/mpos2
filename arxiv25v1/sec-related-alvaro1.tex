
%%%%%%%%%%%%%%%%%%%%%%%%%%%%%%%%%%%%%%%%%%%%%%%%%%%%%%%%%%%%%%%%%%%%%%%%%%%%%%%
% Section 6: Related Work
%%%%%%%%%%%%%%%%%%%%%%%%%%%%%%%%%%%%%%%%%%%%%%%%%%%%%%%%%%%%%%%%%%%%%%%%%%%%%%%
\section{Related Work}
\label{sec:related}

Decentralized consensus has evolved from single-leader PoW/PoS chains to leaderless and DAG-based designs. We briefly review key areas:

\subsection{Proof-of-Stake Chain Protocols}
Early PoS protocols (e.g., Snow White~\cite{FC:DaiPasShi19}, Ouroboros Praos~\cite{EC:DGKR18}, Genesis~\cite{CCS:BGKRZ18}) follow a longest-chain model with one leader per slot (chosen via VRFs). These guarantee security under limited throughput. BFT-style PoS systems like Algorand~\cite{Algorand}, HotStuff~\cite{HotStuff}, and Dfinity~\cite{Dfinity} use committees to vote on blocks, but often incur higher communication costs. Hybrid protocols with VDFs~\cite{FC:DebCabTse21} improve unpredictability at the cost of extra delay. Privacy-preserving PoS (Ouroboros Crypsinous~\cite{SP:KKKZ19}) adds anonymity, but is mostly orthogonal to consensus structure. A common challenge is unpredictability: if leaders or committees can be precomputed, adversaries may launch selfish-mining or bribery attacks~\cite{FKLTZ24,rationalattacksPOS}. Our work orthogonalizes these concerns by treating eligibility as a black box (modeled by VRFs or ideal coins).

\subsection{Multi-Proposer and DAG Protocols}
Protocols like PHANTOM/GHOSTDAG~\cite{AFT:SWZ21} and Conflux extend Nakamoto consensus to DAGs by sorting or forking rules. PHANTOM solves a global optimization, while GHOSTDAG greedily orders the DAG. Prism~\cite{Prism} uses multiple chains and a robust ordering layer. Spectre~\cite{EPRINT:MorKulYok18} and Tree-Graph~\cite{EPRINT:ZhaChaLeo18} also weight DAGs. Avalanche~\cite{SnowFamily} achieves probabilistic consensus via repeated subsampled voting on a DAG. Lachesis (Fantom) uses PoS-weighted asynchronous BFT on a DAG to achieve instant finality. These systems show DAGs enable high throughput and finality, but their analyses and assumptions differ. Our \Proj builds on this lineage by using a CCA-weighted fork choice and sliding window, combining ideas of GHOST and GHOSTDAG with stake weighting and finality anchors. We also consider transaction validity and slashing within the DAG, aspects often omitted in earlier DAG designs.

\subsection{Consensus Analysis}
The Bitcoin Backbone framework~\cite{EC:GarKiaLeo15} established CP/CG/CQ for PoW chains; similar analysis applies to PoS chains~\cite{EC:PasSeeShe17}. DAG protocols challenge these proofs because blocks do not lie on a single chain. Recent work (e.g., Fantom's Lachesis analysis, Narwhal-Tusk~\cite{NarwhalTusk}) decouple availability from ordering. Our security proof adapts backbone-style coupling arguments to the DAG setting. We introduce DAG invariants (DG/DQ/DCP/TB) and show that after GST, honest blocks cross-link each other within the window, ensuring one branch's weight monotonically grows. These arguments align with known results (e.g., \cite{EC:GarKiaLeo15} for chain CP, \cite{AFT:SWZ21} for DAG connectivity), but require new steps to handle parallel blocks and CCA choices.

\subsection{Economic and Practical Perspectives}
Multi-proposer designs also impact incentive structure: they can improve short-term censorship resistance and fairness~\cite{FOCIL, multiFee}. Our eligibility and confirmation scheme is compatible with various fee distribution policies. Additionally, implementing VRF eligibility and antichain selection must be done efficiently; \ProjBase's algorithms are designed for practicality (e.g., using greedy references and PRF sampling of transactions).

\paragraph{Future Directions.} We have presented \ProjIdeal and \ProjBase; extending to \ProjFull (with tighter window management and optional long-reference policies) is ongoing work. The ultimate goal is a fully formalized, deployed multi-proposer consensus that rivals chain-based systems in simplicity, while offering superior performance.



\ignore{%%
%%%%%%%%%%%%%%%%%%%%%%%%%%%%%%%%%%%%%%%%%%%%%%%%%%%%%%%%%%%%%%%%%%%%%%%%%%%%%%%
% Section 5: Related Work
%%%%%%%%%%%%%%%%%%%%%%%%%%%%%%%%%%%%%%%%%%%%%%%%%%%%%%%%%%%%%%%%%%%%%%%%%%%%%%%
\section{Related Work}
\label{sec:related}

Decentralized consensus has evolved from single-leader PoW/PoS chains to multi-leader and leaderless designs. The ``blockchain backbone'' approach~\cite{EC:GarKiaLeo15} introduced simple properties (common prefix, chain growth, chain quality) for analyzing longest-chain PoW and PoS protocols under partial synchrony. Follow-up works (Ouroboros~\cite{EC:KRDO17}, Snow White~\cite{CCS:DKLS18}, Bitcoin-NG~\cite{SOSP:EGSvR16}, etc.) improved various aspects like adaptivity, throughput, or responsiveness, but still relied on a single proposer at a time.

To increase throughput, researchers explored relaxing the chain to a directed acyclic graph (DAG) structure. Hashgraph~\cite{BairdHashgraph}, Spectre~\cite{SPECTRE}, and PHANTOM/GHOSTDAG~\cite{AFT:SWZ21} allow many parallel blocks and use novel ordering or voting rules. These typically assume eventual synchrony but not explicit partial synchrony proofs. OHIE~\cite{EC:DipLosPaz18} and Prism~\cite{PODC:BagKwoShi19} formalized DAG-based ledgers with provable security, by effectively running many parallel chains (OHIE uses $O(n)$ chains to achieve $O(n)$ throughput). Those require stronger synchrony or significant overhead.

Another line, \emph{BFT consensus}, for example PBFT~\cite{OSDI:CasLys99}, HotStuff~\cite{HotStuff}, and Algorand~\cite{Algorand}, achieves fast finality under a partially synchronous network but with a known fixed set of participants. They typically incur quadratic or higher message complexity to gather votes on blocks. Hybrid designs like Ethereum's Gasper (casper+ghost) and Polkadot / GRANDPA graft BFT committees onto a longest-chain core to finalize blocks faster.

Multi-leader PoS protocols combining ideas from both domains have been proposed. Avalanche~\cite{SnowFamily} uses randomized sampling and DAG gossip to achieve probabilistic consensus. Conflux~\cite{OSDI:LiLiWan20} orders a DAG via a pivot chain and heavy subgraph pruning. Neither provided a rigorous partially synchronous analysis akin to the backbone.

\Proj builds on these directions: it uses a DAG of blocks, but with a Nakamoto-style backbone analysis (adapted with new DAG properties). The idea of counting ``votes'' via references resembles Inclusive Blockchain~\cite{EC:Gramoli20} and Prism, but our CCA weighting and anchoring are novel. The requirement to bound tips is implicit in protocols like PHANTOM (which assumes a bound on block creation rate relative to network delay to avoid too many tips). Our work explicitly identifies Tip Boundedness as necessary and sufficient (with other properties) for security.

Finally, our fork-choice rule is inspired by Ethereum's GHOST~\cite{GHOST} (which uses subtree sizes) and the CCA rule parallels the idea of a ``diverge point'' in Sprites and DAG-Rider~\cite{DAGRider}. We show that combining local weight comparison (CCA) with a global window anchor yields a robust rule for dynamic availability in PoS.

}%%%%%

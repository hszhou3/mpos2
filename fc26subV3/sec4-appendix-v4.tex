%!TEX root = mainMPOSv1.tex


%%%%%%%%%%%%%%%%%%%%%%%%%%%%%%%%%%%%%%%%%%%%%%%%%%%%%%%%%%%%%%%%%%%%%%%%%%%%%%%%
% Section 4: Cryptareon-Base (NEW)
%%%%%%%%%%%%%%%%%%%%%%%%%%%%%%%%%%%%%%%%%%%%%%%%%%%%%%%%%%%%%%%%%%%%%%%%%%%%%%%%
%\section{Cryptareon-Base: A Practical DAG-Based PoS Protocol}

\section{Supplemental Materials for Section~\ref{sec:base}}
\label{app:base}

\subsection{Parameter discipline}
\label{sec:param-discipline}


To keep the protocol lean and its analysis transparent, we fix the following discipline (all are referenced where used):
\begin{itemize}
  \item \textbf{Window length.} Set the short-reference window to satisfy $w \ge \Delta + \omega(\log \lambda)$. This couples views across partially synchronous delivery and bounds the number of concurrent tips.
  \item \textbf{References and validity.} Each block’s \emph{short references} must form an antichain within the current window $G_{s,w}$. This is the sole validity predicate; we use selection heuristics to reduce redundancy (exact selection is deferred to the appendix).
  \item \textbf{Optional long reference.} An optional \emph{long reference} may be included (weight $0$) to improve connectivity; it does not contribute to fork-choice scoring.
  \item \textbf{Fork choice.} Use a CCA-anchored, window-filtered score over $G_{s,w}$ with deterministic tie-breaking. Equal-weight scoring preserves stake privacy; a stake-weighted variant can be enabled as an ablation.
  \item \textbf{Eligibility.} VRF sortition is domain-separated per slot/chain; only unpredictability and uniqueness are assumed here.
  \item \textbf{Finality depth.} We use the rule “$k$-deep $\Rightarrow$ stable” with $k=\Theta(w)$; constants follow from the proofs in this section.
\end{itemize}
\label{sec:base}




%%%%%%%%%%%%%%%%%%%%%%%%%%%%%%%%%%%%%%%%%%%%%%%%%%%%%%%%%%%%%%%%%%%%%%%%%%%%%%%%
\ignore{%%%%%%%%%%%%%%%%
In Section~\ref{sec:ideal} we analyzed Cryptareon-Ideal, an abstract multi-proposer DAG protocol, and showed that it satisfies the properties (DG, DQ, DCP, TB) and thereby Persistence and Liveness. We now move to Cryptareon-Base, a practical instantiation that relaxes the idealized assumptions to fit a partially synchronous network and proof-of-stake setting. In particular, Cryptareon-Base uses VRF-based eligibility, bounded short references plus an optional long reference, and explicit transaction validity checks. Our task in this section is to demonstrate that these practical constraints preserve the key properties, ensuring that the same soundness guarantees established for the idealized model continue to hold in a realistic environment.

\subsection{Overview}  
\label{subsec:base-overview}

\subsubsection{Assumptions and Instantiation of \ProjIdeal}
\ProjBase is a practical instantiation of \ProjIdeal (Section~\ref{sec:ideal}) under \emph{partial synchrony} with \emph{VRF sortition}, \emph{bounded short references} (window $w$), and an \emph{optional} single long reference for connectivity. Transactions obey UTXO validity; equivocation is slashable. We retain the \emph{anchored, window-filtered} fork-choice and the \emph{CCA-local} conflict rule.

\begin{itemize}
  \item \textbf{Network.} There exists GST after which honest messages are delivered within $\Delta$ slots.
  \item \textbf{Eligibility.} Each validator $v$ is eligible in slot $s$ iff $(y,\pi)=\VRFSign_{\sk_v}(\domsep\|s)$ satisfies $y<p(s,v)$ (Alg.~\ref{alg:block-creation}).
  \item \textbf{References.} In slot $s$, short references must lie in the \emph{window} $W(s;w)=[s-w+1,s]$; at most one \emph{long-ref} $\ell$ to any older block is allowed and carries zero weight.
  \item \textbf{Fork-choice.} We score tips by summing short-reference counts of \emph{descendants} within the last $w$ slots (anchored at a fixed block $a$), with an option to stake-weight contributions (the default equal weights preserve stake privacy) (Alg.~\ref{alg:fork-choice}).
  \item \textbf{Conflicts.} Two incomparable blocks implying conflicting ledger states are resolved by comparing window-filtered branch weights \emph{from their} $c=\CCA(i,j)$ (Alg.~\ref{alg:cca-resolve}).
\end{itemize}

\paragraph{Parameter discipline.} Throughout the analysis we assume
\begin{equation}
\label{eq:base-params}
w \;\ge\; \Delta \;+\; \omega(\log\lambda) \qquad\text{and}\qquad 
H>\tfrac{1}{2}+\varepsilon
\end{equation}
for some constant $\varepsilon>0$. Here $\Delta$ denotes the post-GST network delay bound, and $H$ is the fraction of total stake held by honest parties. The slack $\omega(\log\lambda)$ is used for concentration bounds.

\subsubsection{Protocol Explanation (Honest Player View)}
\label{subsec:base-walkthrough}
We describe the per-slot state machine of an \emph{honest} node $P$; we explicitly indicate where Algorithms~\ref{alg:block-creation}–\ref{alg:cca-resolve} are invoked. 

%(Depending on global numbering these appear as Algorithms~6–9.)

\paragraph{Start of slot $s$.}
$P$ maintains a local DAG $G_s=(V_s,E_s)$ and a ledger state view induced by its \emph{preferred frontier}. It also caches reachability summaries and a per-validator latest-block map.

\begin{enumerate}
  \item \textbf{Eligibility \& transaction sampling.}
  $P$ computes $(\mathsf{ok},\pi,y)\leftarrow\Eligibility(P,s)$. If $\mathsf{ok}=\textsf{true}$, $P$ samples a candidate transaction set $T$ from the mempool (deterministic PRF subsampling seeded by $y$) and prunes conflicts w.r.t.\ current ancestors (UTXO validity).

  \item \textbf{Reference selection (short \& long).}
  $P$ computes a \emph{deterministic} large antichain $R\subseteq V_{s,w}$ of recent blocks (e.g., \GreedyAntichain). If needed for connectivity, it chooses an \emph{optional} long-ref $\ell$ (e.g., last finalized checkpoint). By rule, $|R_{\text{out}}|\le1$ and the long-ref \emph{never} contributes weight.

  \item \textbf{Block creation and gossip (Alg.~\ref{alg:block-creation}).}
  If eligible and after validity checks, $P$ assembles
  \[
  b=\langle 
  \id,\val{=}P,\slot{=}s,\txs{=}T,\refs{=}R\cup\{\ell\},y,\pi,\sigma\rangle,
  \]
  signs it, inserts it locally, and gossips.

  \item \textbf{Block reception (Alg.~\ref{alg:block-reception}, continuous during slot).}
  For each received block $b'$: verify VRF proof and signature; fetch missing parents; check (i) window constraint for short refs, (ii) at most one long-ref, (iii) no cycles, (iv) $R_w$ is an antichain in $G_{s,w}$, and (v) no UTXO conflicts vs.\ ancestors. Reject if any check fails; otherwise integrate and update summaries and $\Tips(G)$.

  \item \textbf{Anchored fork-choice (Alg.~\ref{alg:fork-choice}, end of slot).}
  With anchor $a$ (genesis or last finalized checkpoint), compute for each tip $T\in\Tips(G)$ the window-filtered score
  \[
  W(T)=\sum_{\substack{d\in\Desc^*(T)\\a\leadsto d}}\mathrm{contrib}(d)\cdot 
  \wref(d;w),
  \]
  set preferred frontier $\hat T_s\in\arg\max W(T)$ (tie-break by $(\slot,\id)$), and expose the induced ledger order.

  \item \textbf{Conflict resolution via CCA (Alg.~\ref{alg:cca-resolve}, on demand).}
  When two incomparable heads $i,j$ imply conflicting states, $P$ computes $c=\CCA(i,j)$ and compares $\BranchW(i;c,w)$ vs.\ $\BranchW(j;c,w)$. The winner determines which subtree remains eligible for inclusion in the preferred branch. This decision is \emph{consistent} with the anchored fork-choice (the preferred frontier must lie on the heavier branch past the CCA).
\end{enumerate}


}%%%%%%%%
%%%%%%%%%%%%%%%%%%%%%%%%%%%%%%%%%%%%%%%%%%%%%%%%%%%%%%%%%%%%%%%%%%%%%%%%%%%%%%%%





\subsection{Protocol Details}
\label{subsec:base-details}

\subsubsection{Notation and Objects}
\label{subsec:notation}
We reuse the notation from Section~\ref{sec:ideal}. In addition:
\paragraph{Ledger Validity and Conflicts.}
Let $S$ be the current ledger state. A multiset of transactions $T\subseteq \Tx$ is valid if $\Valid(S,T)=\textsf{true}$. Two transactions $x,y$ conflict (denoted $\Conflicts(x,y)$) if they cannot both appear in any valid ledger state (e.g., they double-spend a UTXO). For a set of blocks $B\subseteq V$, let $\txs(B)=\bigcup_{b\in B}\txs(b)$ be the union of their transactions. We say a set $T$ conflicts with $B$, written $\Conflicts(T,B)$, if any transaction in $T$ conflicts with any in $\txs(B)$, or any two transactions in $T$ conflict with each other.

\paragraph{Reference Window and Long-Refs.}
Fix a window parameter $w\in\mathbb{N}$. In slot $s$, the reference window is $W(s;w)=\{t: s-w+1\le t \le s\}$. Let $V_{s,w}=\{v\in V: \slot(v)\in W(s;w)\}$ and $E_{s,w}=\{(u\to v)\in E: u,v\in V_{s,w}\}$. Write $G_{s,w}=(V_{s,w},E_{s,w})$. For any block $b$ at slot $s$, partition its references as short vs. long:
\begin{equation*}
\shortrefs(b)=\{r\in\refs(b): s-\slot(r)<w\}, \qquad
\longrefs(b)=\refs(b)\setminus\shortrefs(b).
\end{equation*}
We require $|\longrefs(b)|\le1$. If there is a long-ref, call it $\longref(b)\in\Id$, and it contributes zero weight in fork-choice.

\paragraph{VRF-Based Eligibility (PoS).}
Let $(\VRFSign,\VRFVerify)$ be a secure VRF. In slot $s$, validator $v$ computes $(y,\pi)=\VRFSign(\sk(v);\ \domsep\|\Encode(s))$ with $y\in[0,1)$. Let $p(s,v)$ be the per-slot success probability (typically $p(s,v)=\phi\cdot\stakefrac{v}$ for a global parameter $\phi$). Then define
\begin{equation*}
\mathsf{ok} = (y < p(s,v)),\quad \Eligibility(v,s)\text{ outputs }(\mathsf{ok},\pi,y).
\end{equation*}
On seeing a block $b$ with $(y(b),\pi(b))$, nodes verify the VRF proof $$\VRFVerify(\pk(v);\allowbreak \domsep\|\Encode(\slot(b)), y(b), \pi(b))$$ and a digital signature $\VerifySig(\pk(v); b, \sigma(b))$. If either check fails, the block is ignored.


\subsubsection{Block Creation (\ProjBase)}
\begin{algorithm}[htbp!]
\caption{Block Creation (\ProjBase)}
\label{alg:block-creation}
\begin{algorithmic}[1]
\Require Ledger state $S$, mempool $M$, validator $v$, key $\sk(v)$, stake $\stake(v)$, slot $s$, window $w$
\Ensure New block $b$ or $\bot$ if ineligible
\State $(\mathsf{ok}, \pi, y) \gets \Eligibility(v,s)$
\If{not $\mathsf{ok}$} \Return $\bot$ \EndIf
\State $T \gets \SelectTxs(M; \rho = \Hash(y\|s))$
\Comment{Default: take up to $M_{\max}$ valid txs. \\
Optional: subsample using seed $\rho$ for redundancy reduction.}
\State $R \gets \GreedyAntichain(G_{s,w})$
\Comment{Deterministically select large antichain of recent blocks}
\State $\ell \gets \OptionalLongRef(G, s, w, R, T)$
%\n% --- THIS PART IS NEW: long-ref trigger policy (weight zero in fork-choice) ---\n% 
Trigger: if reachability summary detects a new weakly-connected component outside $[s-w+1,s]$, set $\ell$ to its highest checkpoint; otherwise $\ell\gets\bot$. Long-refs carry zero weight in fork-choice.
%\n% --- END NEW ---
\Comment{Optional single long-ref (e.g., last checkpoint)}
\If{$\Conflicts(T, \Anc^*(R\cup\{\ell\}\setminus\{\bot\}))$}
  \State prune a minimal subset of $T$ to remove conflicts, or \Return $\bot$
\EndIf
% --- THIS PART IS NEW: enforce antichain-closure over refs (no ancestor/descendant pairs) ---\n
\State \textbf{assert} $\forall r\neq r'\in R\cup\{\ell\}\setminus\{\bot\}:\ r\notin \Anc^*(r')\land r'\notin \Anc^*(r)$%\n
% --- END NEW ---\n
\State $\refs \gets 
\begin{cases}
R \cup \{\ell\} & \text{if }\ell\neq\bot,\\
R & \text{otherwise.}
\end{cases}$
\State $\id \gets \Hash(v \,\|\, s \,\|\, \refs \,\|\, \Merkle(T) \,\|\, y \,\|\, \pi)$
\Comment{Define block identifier deterministically}
\State $\sigma \gets \Sign(\sk(v); \id)$
\State $b \gets \langle \id, v, s, T, \refs, y, \pi, \sigma\rangle$
\State $\IntegrateAndGossip(b)$
\State \Return $b$
\end{algorithmic}
\end{algorithm}

\paragraph{Slashing pointer.} Slashable evidence is supported; see App.~\ref{app:slashing-details} for details.




\paragraph{Details:} $\SampleMempool(M,1/N_{\MCP};\rho)$ returns a pseudo-random subset of transactions from $M$ using seed $\rho$, to ensure verifiability. For example, setting the fraction to $1/N_{\MCP}$ partitions $M$ across concurrent proposals, reducing duplication. This sampling rate can be tuned for more redundancy if desired. $\GreedyAntichain(G_{s,w})$ deterministically selects a large antichain in $G_{s,w}$. $\OptionalLongRef$ optionally returns a single ancestor (e.g., the last finalized block) outside the window, for connectivity. We ensure $\txs(b)$ is conflict-free with existing ancestor transactions, and we define $\id(b)$ as the hash of $(\val(b),\slot(b),\txs(b),\refs(b),y(b),\pi(b))$ (this $\id$ is then signed to produce $\sigma(b)$). The new block $b$ is then integrated into the local DAG and gossiped.

\subsubsection{Block Reception and Integration}
\begin{algorithm}[htbp!]
\caption{Block Reception (\ProjBase)}
\label{alg:block-reception}
\begin{algorithmic}[1]
\Require Incoming block $b$, local DAG $G=(V,E)$, current slot $s$
\Ensure $b$ is accepted into $G$ or deferred/rejected
\If{$b\in V$} \Return \EndIf
\If{$\exists r\in \refs(b)$ such that $r\notin V$}
   \ForAll{$x\in \refs(b)$ \textbf{with} $x\notin V$}
      \State request missing parent $x$
   \EndFor
   \State enqueue $b$ \Comment{defer until parents arrive}
   \State \Return
\EndIf
\State $(\mathsf{ok}',\pi',y') \gets \VRFVerify(\pk(\val(b));\ \domsep\|\Encode(\slot(b)), y(b), \pi(b))$
\If{\textbf{not} $\mathsf{ok}'$ \textbf{or not} $\VerifySig(\pk(\val(b)); b, \sigma(b))$} 
  \State \textbf{reject} $b$ \Comment{VRF proof or signature invalid}
\EndIf
\If{$y(b) \ge p(\slot(b),\val(b))$}
   \State \textbf{reject} $b$ \Comment{VRF output above threshold (slashable offense)}
\EndIf
\If{$\exists r\in \refs(b)$ such that $\slot(b)-\slot(r) \ge w$} 
  \State \textbf{reject} $b$ \Comment{short ref outside window}
\EndIf
\If{$|\{r\in \refs(b): \slot(b)-\slot(r) \ge w\}| > 1$}
  \State \textbf{reject} $b$ \Comment{more than one long-ref used}
\EndIf
\If{$\exists r\in \refs(b)$ such that $r\in \Desc^*(b)$}
  \State \textbf{reject} $b$ \Comment{cycle detected}
\EndIf
\If{$\exists r,r'\in \refs(b),\ r\neq r'$ such that $r\nparallel r'$ in $G$}
  \State \textbf{reject} $b$ \Comment{$R_w$ not an antichain ($\parallel$ denotes incomparability)}\mnote{This the second thing I wanted to highlight. We might need to check if the descendants refs are also incomparable. It would be good to eliminate a situation like this ($a \to c$ and $a \to b \to c$) and it's generalization ($a \to x$ and $a \to b \to ... \to x$)}
\EndIf
\If{$\Conflicts(\txs(b), \Anc^*(\refs(b)))$}
  \State \textbf{reject} $b$ \Comment{UTXO conflict with ancestor}
\EndIf
\State Insert $b$ and edges $(r \to b)$ for all $r\in\refs(b)$ into $G$
\State Update $\Tips(G)$, reachability summaries, and latest-block map $L$
\State $\ForkChoiceUpdate(G)$
%\State $\Broadcast(b)$
\end{algorithmic}
\end{algorithm}

\begin{remark}
The integration logic assumes one-slot network delays for simplicity. In practice, a node might receive a block with $\slot(b) < s$. The checks still apply; if $s-\slot(b) < w$, the block can be integrated (it arrived late but within the window).
\end{remark}

\subsubsection{Fork Choice and Conflict Resolution (\ProjBase)}
\begin{algorithm}[htbp!]
\caption{Anchored Fork Choice (\ProjBase)}
\label{alg:fork-choice}
\begin{algorithmic}[1]
\Require Anchor $a$, $\Tips(G)$, window $w$
\Ensure preferred frontier $\hat{T}$
\ForAll{$d\in V$ such that $a\leadsto d$}
  \State $\mathsf{score}(d) \gets 0$
\EndFor
\ForAll{$d\in V$ such that $a\leadsto d$}
  \State $\mathsf{score}(d) \mathrel{+}= \sum_{r\in \shortrefs(d)} \contrib(d)$ \Comment{$\contrib(d)=1$ by default; $=\stakefrac{\val(d)}$ if stake-weighted}
  \State $\mathsf{score}(d) \mathrel{+}= \mathsf{score}(\text{parent of }d\text{ in }G)$ \Comment{like DFS traversal or DP}
\EndFor
\State $M \gets \max_{T\in \Tips(G):\,a\leadsto T} \mathsf{score}(T)$
\State $\hat{T} \gets \{ T\in \Tips(G) : a\leadsto T,\ \mathsf{score}(T) = M \}$
\State \Return $\hat{T}$
\end{algorithmic}
\end{algorithm}

\begin{remark}
We assume each validator stores or can compute $\mathsf{score}(x)$ for each $x\in V$ on its preferred branch. Practically, one can maintain $\mathsf{score}$ values incrementally as new blocks arrive.
\end{remark}

\begin{algorithm}[htbp!]
\caption{Conflict Resolution (\ProjBase)}
\label{alg:cca-resolve}
\begin{algorithmic}[1]
\Require Conflicting blocks $i,j$ (detected UTXO conflict), window $w$
\Ensure Winner $h^*$
\State $c \gets \CCA(i,j)$
\State $W_i\gets\BranchW(i;c,w)$, \quad $W_j\gets\BranchW(j;c,w)$
\If{$W_i > W_j$} \Return $i$
\ElsIf{$W_j > W_i$} \Return $j$
\Else 
      \If{$y(i) < y(j)$} \Return $i$
      \Else \Return $j$ \Comment{tie-break by VRF output}
      \EndIf
\EndIf
\end{algorithmic}
\end{algorithm}

\begin{remark}
Two conflicting transactions on different branches trigger the above resolution. The loser branch is pruned from the ledger view (kept in the DAG), but is not permanently discarded until the honest branch reaches finality depth $k$.
\end{remark}

\paragraph{Finalization heuristic.}
$P$ may \emph{finalize} a block $B$ once it is $k$-deep under $\hat T_s$, with $k=\Theta(w)$ as derived in Theorem~\ref{thm:base-safety}. (Finalization is monotone and checkpoint-compatible.)




\subsection{Security of the Practical Protocol: Details}
\label{app:base-security}
We prove the DAG properties and ledger properties for \ProjBase under~\eqref{eq:base-params}. We write $\lambda_h \triangleq \mathbb{E}[\#\text{honest eligible per slot}]=\Theta(pH|\Val|)$.


\subsubsection{Consensus properties: DG, DQ, TB, DCP}
\begin{theorem}[Consensus properties of \ProjBase]\label{thm:base-consensus-summary}
Under the parameter discipline $w \ge \Delta + \omega(\log \lambda)$ and $H>1/2+\varepsilon$,
\ProjBase{} satisfies DG, DQ, TB, and DCP after GST with constant parameters (in~$\lambda$).
Formal statements and proofs are given in Appendix~\ref{app:base-security}.
\end{theorem}


\begin{theorem}[DAG Growth (DG) in \ProjBase]
\label{thm:base-DG}
For any interval of $\ell$ slots after GST, at least $\tau_D\ell$ honest blocks become ancestors of some honest tip by time $s{+}\ell{-}1{+}\Delta$, with probability $1-\negl(\lambda)$, for $\tau_D=\Theta(pH)$.
\end{theorem}
\begin{proof}[Proof sketch]
Post-GST, in each slot $s$ the number of honest eligible is $\mathrm{Bin}(|\Val|,p\cdot \stakefrac{\cdot})$; Chernoff concentration gives $\Omega(\lambda_h)$ honest blocks per slot over $\ell$ slots. Delivery within $\Delta$ ensures these blocks are referenced by subsequent honest blocks inside the window $w\ge \Delta$, hence they become ancestors of honest tips.
\end{proof}

\begin{theorem}[DAG Quality (DQ) in \ProjBase]
\label{thm:base-DQ}
Fix any $\ell\ge \Omega(\log\lambda)$ after GST. The fraction of honest blocks among those entering the preferred frontier during the interval is at least $\mu_D=\Theta(H)$ w.h.p.
\end{theorem}
\begin{proof}[Proof sketch]
VRF eligibility is stake-proportional; the expected honest fraction per slot is $H$. Window-filtered anchored scoring does not amplify adversarial inputs beyond the window; by linearity and concentration over $\ell$ slots the realized fraction stays within a small deviation of $H$.
\end{proof}

\begin{theorem}[Tip Boundedness (TB) in \ProjBase]
\label{thm:base-TB}
Assume $w\ge \Delta$. Then $\mathrm{TB}[\beta]$ holds with $\beta=O(\lambda_h\cdot \Delta)$ w.h.p.
\end{theorem}
\begin{proof}[Proof sketch]
Within any $\Delta$-slot horizon, honest blocks produced without mutual visibility may introduce multiple tips. After at most $\Delta$ slots these blocks are delivered and subsequent honest blocks (in slots $>s{+}\Delta$) reference \emph{all} visible recent tips within the window, merging them. Hence the tip set size is stochastically dominated by the number of honest blocks created within the last $\Delta$ slots, i.e., $O(\lambda_h\Delta)$, up to lower-order adversarial contributions that are also merged once visible.
\end{proof}

\begin{theorem}[DAG Common Past (DCP) in \ProjBase]
\label{thm:base-DCP}
There exists $k_D=\Theta(\Delta)$ such that, after removing the last $k_D$ layers under the preferred frontiers of any two honest parties, the remaining ancestor sets coincide w.h.p.
\end{theorem}
\begin{proof}[Proof sketch]
By TB, the honest frontier has bounded width $O(\lambda_h\Delta)$. After $\Delta$ slots, all honest blocks from earlier slots are visible to all honest parties and get cross-referenced within the window, causing the CCA between any two preferred frontiers to move forward monotonically. Trimming the last $\Theta(\Delta)$ layers removes the area where asynchrony can cause transient divergence, leaving an identical common past.
\end{proof}


\subsubsection{Ledger properties: Persistence and Liveness}
\begin{theorem}[Persistence (Safety) of \ProjBase]
\label{thm:base-safety}
Let $k=\Theta(w)$ be chosen so that in $w$ slots the honest branch accumulates a fixed constant margin of window-filtered weight against any competing branch past the same CCA. If a block $B$ is $k$-deep under an honest party's preferred frontier at time $t$, then with probability $1-\negl(\lambda)$, $B$ remains in every honest party's ledger at all later times.
\end{theorem}
\begin{proof}[Proof sketch]
Consider a conflicting head $j$ and honest head $i$ with $c=\CCA(i,j)$. By DG, DQ, and TB, during any $w$-slot window after $c$ the honest branch contributes $\Omega(\lambda_h w)$ window-filtered references; the adversary contributes $O(\alpha |\Val| p w)$ \emph{only} to blocks \emph{actually created} within the window (withheld older references carry zero weight). Since $H>\tfrac{1}{2}{+}\varepsilon$, the honest margin in the window is positive with high probability and grows with $w$. After $k=\Theta(w)$ depth, any tie can no longer be reached by adversarial scheduling; anchored fork-choice and CCA resolution keep selecting the honest branch, preserving $B$.
\end{proof}

\begin{theorem}[Liveness of \ProjBase]
\label{thm:base-liveness}
For any valid transaction $\tx$ broadcast after GST, there exists $\ell_{\mathrm{live}}=\tilde{O}\!\left( \frac{1}{pH} + \Delta \right)$ such that $\tx$ is included in the ledger of all honest parties within $\ell_{\mathrm{live}}$ slots, except with negligible probability.
\end{theorem}
\begin{proof}[Proof sketch]
With probability $\Omega(pH)$ per slot, some honest validator is eligible and can include $\tx$. Within $\Delta$ slots, the block diffuses to all honest parties; within additional $O(w)$ slots (with $w\ge \Delta$) subsequent honest blocks reference and bury it. The anchored, window-filtered fork-choice maintains the branch as preferred (Theorem~\ref{thm:base-safety}), so $\tx$ propagates to all honest ledgers within $\tilde{O}(1/(pH)+\Delta)$.
\end{proof}

\subsubsection{Coupling to the Idealized Analysis}
\label{subsec:base-coupling}
\ProjBase is a refinement of \ProjIdeal: unbounded referencing is replaced with \emph{window-bounded} short references plus an \emph{optional} long-ref; public coins are realized with VRFs; synchrony is relaxed to partial synchrony with $w\ge \Delta$. The proofs above adapt the arguments of Section~\ref{sec:ideal} by (i) adding $\Delta$-slot delivery delays in the TB and DCP lemmas; (ii) ensuring that window-filtered weights ignore stale withheld work; and (iii) using stake-proportional VRF concentration for DQ. Thus, each ideal lemma has a practical analogue with slightly weaker (but still constant) parameters.

\subsubsection{Design Soundness: Remarks and Additional Concerns}
\label{subsec:base-remarks}
\begin{remark}[Window size vs.\ delay]
If $w<\Delta$, honest tips can \emph{expire} from the short-reference window before cross-linking, violating TB and undermining DCP and safety. Hence the constraint $w\ge \Delta$ is \emph{necessary}.
\end{remark}

\begin{remark}[Optional long-ref policy]
Long-refs guarantee global connectivity but carry zero weight. Implementations should \emph{deterministically} choose $\ell$ (e.g., last finalized checkpoint or the $\prec$-max ancestor of the smallest short-ref) to avoid adversarial steering and to enable fast parent fetching.
\end{remark}

\begin{remark}[Slashing conditions]
If a validator is caught producing two blocks in the same slot (equivocation) \emph{or} a block in a slot for which it was not eligible ($y(b)\not< p(\slot(b),\val(b))$), a slashable fault occurs. In \ProjBase, such misbehavior can be detected via conflicting signatures and VRF proofs. An honest observer can include a special transaction containing $(\id(b),\id(b'),y(b),y(b'),\pi(b),\pi(b'),\sigma(b),\sigma(b'))$ to slash $v$'s stake. This disincentivizes equivocation and threshold-faking (even short-range attempts yield no lasting gain, as they cannot increase branch weight beyond the current window).
\end{remark}

\subsubsection{Discussion of Security and Parameters}
\noindent
\textbf{Safety:} The combination of antichain reference rules, UTXO conflict checks, and the CCA-anchored fork choice ensures that once honest blocks gain a majority of in-window weight over any adversarial branch, the honest branch will prevail. The defined properties (DG/DQ/DCP/TB) hold with high probability under the usual honest majority assumptions and appropriate parameter choices, leading to persistence.

\smallskip

\noindent
\textbf{Liveness:} By allowing multiple proposers, the protocol increases the rate at which honest votes (references) are cast, helping convergence even with latency. The sliding window (with $w$ sufficiently large relative to $\Delta$) absorbs network delays. Occasional long-refs ensure the DAG does not fragment. Thus, any valid transaction will eventually be included and confirmed.

\smallskip

\noindent
\textbf{Resistance to Withholding:} Because we only count in-window references anchored at the CCA, an adversary cannot secretly build a long private branch that overtakes the honest public branch upon revelation. Withholding yields no advantage once honest blocks keep referencing the honest tips.

\smallskip

\noindent
\textbf{Parameter Choices:} A window on the order of tens of slots (e.g., $w\approx 30$) typically suffices to cover network latency for a moderate honest block rate. The parallelism factor $N_{\MCP}$ tunes how many transactions each block samples, balancing leaderless throughput with verifier load.

\subsection{Cross-Section Correspondence (Sections \ref{sec:ideal} vs \ref{sec:base})}
The following summary illustrates how the idealized protocol maps to \ProjBase:
\begin{itemize}
  \item \textbf{Eligibility:} Ideal uses implicit Bernoulli coins per slot; Base uses VRF-based eligibility (see Algorithms~\ref{alg:ideal-create}, \ref{alg:block-creation}).
  \item \textbf{Antichain selection:} Ideal selects an exact max antichain (Alg.~\ref{alg:ideal-antichain}); Base uses a greedy antichain within the window (or exact if feasible).
  \item \textbf{Long reference:} Ideal has none ($\OptionalLongRef$ always returns $\bot$); Base allows one deterministic long-ref outside the window.
  \item \textbf{Block creation:} Ideal ignores dependencies and VRFs (Alg.~\ref{alg:ideal-create}); Base includes VRF proof, UTXO conflict checks, and optional long-ref (Alg.~\ref{alg:block-creation}).
  \item \textbf{Block reception:} Ideal only checks acyclicity and parent presence (Alg.~\ref{alg:ideal-reception}); Base verifies VRF/signature, window/antichain constraints, and UTXO validity (Alg.~\ref{alg:block-reception}).
  \item \textbf{Fork choice:} Ideal uses anchored fork choice with unit weights (Alg.~\ref{alg:ideal-anchored-fc}); Base optionally stake-weights references and anchors to finalized checkpoints (Alg.~\ref{alg:fork-choice}).
  \item \textbf{Conflict resolution:} Ideal and Base use the same CCA-based rule (Alg.~\ref{alg:ideal-cca} vs \ref{alg:cca-resolve}), with Base optionally stake-weighting.
\end{itemize}

To conclude, Cryptareon-Base inherits the security guarantees of Cryptareon-Ideal: under partial synchrony and honest majority, it satisfies DAG Growth, DAG Quality, DAG Common Past, and Tip Boundedness, and hence achieves Persistence and Liveness. Having established these theoretical guarantees, we now turn to empirical validation. Section~\ref{sec:eval} presents our implementation and simulation results, where we measure reorg distributions, tip counts, and finality times under different network delays and adversarial stake. These experiments serve to corroborate the analytical bounds, illustrating how the properties manifest in practice and confirming the protocol’s robustness under realistic conditions.





%%%%%%%%%%%%%%%%%%%%%%%%%%%%%%%%%%%%%%%%%%%%%%%%%%%%%%%%%%%%%%%%%%%%%%%%%%%%%%%%
\ignore{%%%%%%%%%%%%%%%%



}%%%%%%%%%%%%%%%%%%%%%
%%%%%%%%%%%%%%%%%%%%%%%%%%%%%%%%%%%%%%%%%%%%%%%%%%%%%%%%%%%%%%%%%%%%%%%%%%%%%%%%

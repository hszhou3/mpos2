%!TEX root = mainMPOSv1.tex

%%%%%%%%%%%%%%%%%%%%%%%%%%%%%%%%%%%%%%%%%%%%%%%%%%%%%%%%%%%%%%%%%%%%%%%%%%%%%%%
% Section 2: Security Model/Definitions 
%%%%%%%%%%%%%%%%%%%%%%%%%%%%%%%%%%%%%%%%%%%%%%%%%%%%%%%%%%%%%%%%%%%%%%%%%%%%%%%



\section{Security Model} \mnote{This section only is 9 pages, we must be brutal with it...}
\label{sec:model}

We adopt a standard cryptographic model for blockchain protocols~\cite{EC:GarKiaLeo15,EC:PasSeeShe17,JC:Canetti00}. The system comprises a set of validators $\cP$ and an environment $\cZ$ controlling inputs, timing, and corruptions. Time is divided into discrete slots $s=1,2,\ldots$. We assume a partially synchronous network~\cite{DLS88,EC:PasSeeShe17}: after some unknown Global Stabilization Time (GST), all honest messages are delivered within a known bound $\Delta$; before GST, the adversary controls delays. 

Players start from a common genesis block containing initial public keys and stake distribution. Validators may join or leave, and stake may change over time; for simplicity we consider a “flat” model where any player can register multiple identities to simulate more stake. The adversary $\cA$, controlled by $\cZ$, may adaptively corrupt validators up to a fraction $\alpha$ of total stake $\StakeTot=\sum_{v\in\cP}\stake(v)$. We assume $\alpha < 1/2$ unless otherwise specified. Upon corruption, $\cA$ learns the validator's secret key and internal state. Honest parties erase state upon leaving (non-persistent corruption). \mnote{The corruption is not directly used, maybe be removed?}

The network is adversarial: the adversary $\cA$ controls message scheduling, reordering, and can delay honest messages (subject to the $\Delta$ bound after GST) but cannot modify or forge them. Any message broadcast by an honest party reaches all honest parties within at most $\Delta$ slots (after GST). \mnote{This adversary is not really used in our setting -- it's too powerful. Delete?}

\paragraph{VRF-based Eligibility.} Each validator $v$ holds a key pair $(\pk_v,\sk_v)$ and stake $\stake(v)$. In slot $s$, validator $v$ computes a VRF with domain separation:
\begin{equation*}
(y,\pi) = \VRFSign_{\sk_v}\bigl(\domsep \,\|\,\Encode(s)\bigr),
\end{equation*}
where $\domsep$ is protocol-specific constant, and $\Encode$ is a canonical encoding. We assume the VRF is secure: its output $y$ is unpredictable and unbiasable. Let $p(s,v)\in[0,1]$ be the per-slot threshold (e.g., $p(s,v) = \phi\cdot \stakefrac{v}$ for a global factor $\phi$ controlling expected parallelism). If $y < p(s,v)$, then $v$ is eligible to propose a block in slot $s$. In that case, $v$ produces a VRF proof $\pi$ with the block.

\paragraph{Block and DAG Structure.} When eligible, validator $v$ may create a block
\begin{equation*}
b = \langle \id(b),\slot(b),\val(b)=v,\txs(b),\refs(b),\longref(b),y(b),\pi(b),\sigma(b)\rangle,
\end{equation*}
where:
\begin{itemize}
  \item $\id(b)$ is its unique identifier (a hash),

  \item $\slot(b)=s$ is its slot and $\val(b)=v$ is the creator,
  \item $\txs(b)$ is a set of transactions (e.g., UTXOs) chosen from the mempool,
  \item $\refs(b)$ is a set of hashes of parent blocks, all within the recent window $[s-w+1,s]$ (“short references”),
  \item $\longref(b)$ is an optional single reference to an older block (outside the short window) for connectivity,
  \item $y(b)$ and $\pi(b)$ are the VRF output and proof, included in the block,
  \item $\sigma(b)$ is a signature on the block contents by the validator $v$.
\end{itemize}
The validator must ensure $\txs(b)$ is conflict-free (no double-spends) relative to the ledger state, and $\refs(b)\cup\{\longref(b)\}$ is an antichain in the DAG, to preserve consistency. Honest blocks are immediately gossiped to all nodes. The union of all blocks and reference edges forms a directed acyclic graph (DAG) $G$. We write $u\leadsto v$ if there is a directed path from block $u$ to $v$ in $G$. For a block $b$, let $\Anc(b)=\{a: a\leadsto b\}$ and $\Desc(b)=\{d: b\leadsto d\}$, with $\Anc^*(b)=\Anc(b)\cup\{b\}$, $\Desc^*(b)=\Desc(b)\cup\{b\}$. Two blocks $u$ and $v$ are {\em incomparable} (denoted $u\parallel v$) if neither $u\leadsto v$ nor $v\leadsto u$.

\paragraph{The Execution of Protocol.} The execution of the protocol is an interactive process: in each slot $s$, each honest player collects all messages (with adversarial delay $\le\Delta$), checks eligibility via VRF, possibly creates a block, and broadcasts it. The adversary may inject blocks for corrupted players and deliver or withhold messages. We denote by $\exec_{\Pi,\cA,\cZ}$ the complete execution transcript. Let $\chainset^s$ be the set of all chains (paths from genesis) visible to honest parties at slot $s$, and $\chain_i^s$ be the chain of party $i$ at slot $s$.

\subsection{Consensus Security Properties}
\label{sec:DG-DQ-DCP}

We recall the canonical properties for chain protocols~\cite{EC:GarKiaLeo15}: {\em Common Prefix (CP[$k$])} ensures any two honest chains differ only in the last $k$ blocks; {\em Chain Growth (CG[$\tau,\ell$])} guarantees any honest chain grows by at least $\tau\ell$ blocks over any $\ell$-slot window; and {\em Chain Quality (CQ[$\mu,\ell$])} ensures that in any $\ell$ consecutive blocks of an honest chain, at least a $\mu$ fraction are from honest parties. These imply persistence and liveness for classical blockchains~\cite{EC:GarKiaLeo15}.
For DAG-based protocols, we define analogous properties. 


\begin{figure}[htbp!]
\centering
\begin{tikzpicture}[x=1cm,y=0.8cm,>=Latex,thick]

  % Black = blocks
  \fill[black] (0,0) circle (2pt);
  \node[right=0.4cm of {(0,0)}] {\scriptsize Generic block};

  % Green = honest/finalized prefix
  \fill[green!40!black] (0,-1) rectangle (0.4,-0.6);
  \node[right=0.6cm of {(0.4,-0.8)}] {\scriptsize Honest/finalized prefix};

  % Blue = frontier
  \fill[blue!30] (0,-2) rectangle (0.4,-1.6);
  \node[right=0.6cm of {(0.4,-1.8)}] {\scriptsize Frontier / last $k_D$ layers};

  % Gray = common past
  \fill[black!20] (0,-3) rectangle (0.4,-2.6);
  \node[right=0.6cm of {(0.4,-2.8)}] {\scriptsize Common past region};

  % Red = transaction broadcast
  \fill[red!70] (0,-4) circle (2pt);
  \node[right=0.4cm of {(0,-4)}] {\scriptsize Transaction broadcast / inclusion};

\end{tikzpicture}
\caption{\textbf{Legend of colors used in Figures~\ref{fig:dg}--\ref{fig:liveness}.} 
Black = blocks; green = honest or finalized prefix; 
blue = frontier region; gray = common past; red = broadcast transaction event or inclusion.}
\label{fig:legend}
\end{figure}




\subsubsection{DAG Growth (DG)}
\label{sec:DG}

In any window of $\ell$ slots after GST, honest work makes \emph{linear progress along some honest chain} in the DAG: at least $\tau_D\cdot \ell$ honest blocks created in that window end up on the past of an honest party’s fork-choice tip (i.e., on its preferred branch).

\begin{figure}[htp!]
\centering
\begin{tikzpicture}[x=0.16cm,y=0.6cm,>=Latex,thick]
  % Axis
  \draw[-{Latex}] (0,0) -- (98,0) node[below right] {slots};

  % Interval [s, s+ell]
  \draw[very thick,|-|] (40,0) -- (80,0);
  \node[below] at (40,-0.1) {\scriptsize $s$};
  \node[below] at (80,-0.1) {\scriptsize $s+\ell$};
  \node[below] at (60,-0.2) {\scriptsize length $\ell$};

  % Honest blocks accumulating
  \foreach \x in {42,45,48,51,54,57,60,63,66,69,72,75,78} {
    \fill[black] (\x,0.8) circle (1.1pt);
  }
  \node[align=left] at (16,1.1) {\scriptsize Honest blocks created in $[s,s+\ell]$ \\[-1pt] \scriptsize join the past of some honest preferred tip};

  % Arrow up
  \draw[->] (60,0.6) -- (60,1.3);
  \node[above] at (60,1.3) {\scriptsize DG: $\ge \tau_D\ell$ honest on a preferred branch (w.h.p.)};
\end{tikzpicture}
\caption{\textbf{DAG Growth (DG).}
\small
Over any $\ell$-slot interval after GST, with probability $1-\negl(\lambda)$ at least $\tau_D\ell$ honest blocks become ancestors of \emph{an honest party’s fork-choice tip} (i.e., lie on some honest preferred branch). 
Black dots represent honest blocks produced during the interval.}

\label{fig:dg}
\end{figure}

\begin{definition}[DAG Growth]\label{def:DG}
Fix a security parameter $\lambda$ and let $\negl(\cdot)$ denote a negligible function.
Let $\mathrm{GST}$ be the global stabilization time and $\Delta$ the post-GST delay bound.
For any interval $I=[s,s+\ell-1]$ of $\ell$ consecutive slots with $s\ge \mathrm{GST}$, define $H_I$ to be the set of honest blocks created in $I$ that, by the end of slot $s+\ell-1+\Delta$, are contained in the ancestor set of the fork-choice tip of at least one honest party (equivalently, they lie on some honest party’s preferred branch).
We say $\mathrm{DG}[\tau_D,\ell]$ holds if for all such intervals
\[
\Pr\!\big[\,|H_I|\ \ge\ \tau_D\cdot \ell\,\big]\ \ge\ 1-\negl(\lambda).
\]
\end{definition}

\begin{remark}[Why this formulation]
In a leaderless multi-proposer DAG, many honest tips can coexist and grow in parallel.
Requiring linear growth of \emph{all} tips would conflate concurrency with lack of progress.
The DG we define here focuses on \emph{linear progress along some honest chain} (the past of an honest fork-choice tip), which is the DAG analogue of chain-growth in linear blockchains: despite parallelism, honest parties make steady ledger progress on a preferred branch.
This avoids interpreting DG as a global linearization of the whole DAG while still providing the liveness backbone needed for the safety/liveness proofs.
\end{remark}

% ============================= THIS PART IS NEW: Sept 11, 2025=============================
\paragraph{DAG Linearization.}
Fix a total order $\prec$ on block identifiers (e.g., lexicographic hash), and an \emph{anchor} $a$
(genesis or the last finalized checkpoint). For any honest view $G_t$ at slot $t$, define the
\emph{anchored linearization} $\Lin_t \equiv \Lin_a(G_t,\prec)$ to be the (unique) total order obtained by
taking any topological sort of the sub-DAG $\Anc^*(\Tips(G_t)) \cap \Desc^*(a)$ and breaking ties by
$(\text{slot},\text{id})$ according to~$\prec$. This linearization is exactly the ledger order induced by the
anchored fork-choice (up to deterministic tie-breaking).

\paragraph{DG (linearized reformulation).}
Let $I=[s,s+\ell-1]$ be any $\ell$-slot interval after GST and let $H_I$ be the set of honest blocks
created in $I$. By time $t=s+\ell-1+\Delta$, consider $\Lin_t$. We say \textbf{DG-Lin[$\tau_D,\ell$]}
holds if, except with negligible probability,
\[
\big|\{\,b\in H_I : b \text{ appears in } \Lin_t\,\}\big| \ \ge\ \tau_D\cdot \ell .
\]
Intuitively: when we read the DAG through its canonical linearization $\Lin_t$, honest work advances at
a linear rate.

\paragraph{Equivalence to the original DG (informal).}
Under our anchored, window-limited fork choice and TB, each honest party’s preferred tip determines a
canonical linearization $\Lin_t$. An honest block that lies on the past of an honest preferred tip (in the 
original DG) \emph{also} appears in $\Lin_t$, and conversely blocks that are absent from all honest preferred
branches cannot appear in the anchored linearization exposed as the ledger. Hence DG and DG-Lin differ
only in presentation; DG-Lin aligns more directly with the DAG’s parallel nature.
% ============================================================================




\subsubsection{DAG Quality (DQ)} 
\label{sec:DQ}

In any window of $\ell$ slots, the fraction of honest blocks among all blocks entering the “recent frontier” (within the last $w$ slots on the preferred branch) is at least $\mu_D$.



\begin{definition}[DAG Quality]\label{def:DQ}
Fix a security parameter $\lambda$. For any interval $I=[s,s+\ell-1]$ with $s\ge \mathrm{GST}$, consider the set $B_I$ of blocks that enter the preferred branch's frontier (i.e., become descendants of the fork-choice tip) during $I$. Let $B_I^{\mathsf{H}}$ be those created by honest parties.
We say $\mathrm{DQ}[\mu_D,\ell]$ holds if
\begin{equation*}
\Pr\big[\,|B_I^{\mathsf{H}}|\ \ge\ \mu_D \cdot |B_I|\,\big]\ \ge\ 1-\negl(\lambda).
\end{equation*}
\end{definition}

\begin{figure}[htp!]
\centering
\begin{tikzpicture}[x=0.16cm,y=0.6cm,>=Latex,thick]
  % Axis
  \draw[-{Latex}] (0,0) -- (98,0) node[below right] {slots};

  % Frontier band over an interval
  \fill[blue!8] (40,1.4) rectangle (80,2.6);
  \draw[blue!60!black] (40,1.4) rectangle (80,2.6);
  \node[blue!60!black] at (60,2.9) {\scriptsize frontier over $[s,s+\ell]$};

  % Honest vs total markers in frontier
  \foreach \x in {42,46,50,54,58,62,66,70,74,78} {
    \fill[black] (\x,2.0) circle (1.1pt); % total
  }
  \foreach \x in {44,48,52,60,68,76} {
    \fill[green!50!black] (\x,2.0) circle (1.6pt); % honest subset
  }

  % Labels s and s+ell
  \draw[very thick,|-|] (40,0) -- (80,0);
  \node[below] at (40,-0.1) {\scriptsize $s$};
  \node[below] at (80,-0.1) {\scriptsize $s+\ell$};

  % Callout
  \node[align=center] at (60,0.9) {\scriptsize DQ: fraction of honest among frontier entrants $\ge \mu_D$ w.h.p.};
\end{tikzpicture}
\caption{\textbf{DAG Quality (DQ).} 
\small 
In any $\ell$-slot window after GST, the fraction of \emph{honest blocks} among those entering the advancing frontier is at least $\mu_D$ with probability $1-\negl(\lambda)$. 
In the illustration, the blue band marks the advancing frontier; 
black dots represent all blocks that enter this frontier, 
and green dots highlight the subset created by honest parties.}
\label{fig:dq}
\end{figure}





\subsubsection{DAG Common Past (DCP)}
\label{sec:DCP}

For any two honest views of the DAG at times $t_1\le t_2$, if we remove the last $k_D$ layers from the current frontier, the remaining past is identical. Equivalently, all honest parties share the set of ancestors up to the most recent CCA of their tips.

\begin{definition}[DAG Common Past (DCP)]\label{def:DCP}
Fix $k_D \in \mathbb{N}$. For an honest party $P$ at slot $t$, let $V_t^P$ denote
its view of the DAG at the end of $t$, and define
$\mathsf{Trim}(V_t^P,k_D)$ as $V_t^P$ with the last $k_D$ layers (by slots)
beneath its fork-choice tip removed.

We say $\mathrm{DCP}[k_D]$ holds if for all honest parties $P,Q$ and all slots
$t_1 \le t_2$ after $\mathrm{GST}$,
\[
\Pr\!\Big[\, \mathsf{Trim}(V_{t_1}^P,k_D)\;\subseteq\;\mathsf{Trim}(V_{t_2}^Q,k_D)\,\Big]
\;\ge\; 1-\negl(\lambda).
\]
That is, once the most recent $k_D$ layers are trimmed, the earlier view of
any honest party is always contained in the later view of any other honest party,
except with negligible probability.
\end{definition}



\begin{figure}[htbp!]
\centering
\begin{tikzpicture}[x=0.16cm,y=0.6cm,>=Latex,thick]
  % Axis
  \draw[-{Latex}] (0,0) -- (98,0) node[below right] {slots};

  % Common past region (identical after trimming)
  \fill[black!8] (0,1.4) rectangle (55,2.6);
  \draw (0,1.4) rectangle (55,2.6);
  \node at (27.5,2.0) {\scriptsize common past (identical)};

  % Frontier windows near t1 and t2 (last k_D layers)
  \fill[blue!8] (50,1.4) rectangle (65,2.6);
  \draw[blue!60!black] (50,1.4) rectangle (65,2.6);
%  \node[blue!60!black] at (62.5,2.95) {\scriptsize $\mathsf{Frontier}_{t_1}^P$};
  \node[blue!60!black] at (57.5,3.1) {\scriptsize $\mathsf{Frontier}_{t_1}^P$};

  \fill[blue!8] (80,1.4) rectangle (95,2.6);
  \draw[blue!60!black] (80,1.4) rectangle (95,2.6);
%  \node[blue!60!black] at (92.5,2.95) {\scriptsize $\mathsf{Frontier}_{t_2}^Q$};
  \node[blue!60!black] at (87.5,3.1) {\scriptsize $\mathsf{Frontier}_{t_2}^Q$};

  % Vertical markers t1, t2
  \draw[dashed] (65,0.3) -- (65,3.2) node[above] {$t_1$};
  \draw[dashed] (95,0.3) -- (95,3.2) node[above] {$t_2$};

  % Braces highlighting "trim" operation
  \draw[decorate,decoration={brace,amplitude=6pt}] (65,2.95) -- (50,2.95)
    node[midway,above=7pt] {\scriptsize remove last $k_D$};
  \draw[decorate,decoration={brace,amplitude=6pt}] (95,2.95) -- (80,2.95)
    node[midway,above=7pt] {\scriptsize remove last $k_D$};

  % Annotation showing subset relation of trimmed DAGs
  \node[align=center] (trimEq) at (40,0.55)
    {\scriptsize $\mathsf{Trim}(V_{t_1}^P,k_D)\;\subseteq\;\mathsf{Trim}(V_{t_2}^Q,k_D)$};

  % Arrows from braces to annotation
  \draw[->,blue!60!black] (57.5,2.65) .. controls (57.5,1.6) and (45,1.2) .. (trimEq.north);
  \draw[->,blue!60!black] (87.5,2.65) .. controls (87.5,1.6) and (55,1.2) .. (trimEq.north);

\end{tikzpicture}
\caption{\textbf{DAG Common Past (DCP).} 
The \textcolor{black!60!white}{gray region} shows the common past that all honest parties agree on. 
The \textcolor{blue!60!black}{blue regions} represent the most recent $k_D$ layers 
(\,$\mathsf{Frontier}_{t_1}^P$, $\mathsf{Frontier}_{t_2}^Q$\,) beneath the fork-choice tips at times $t_1\le t_2$. 
By removing these blue frontier layers, the earlier trimmed DAG is always contained in the later one,
i.e., $\mathsf{Trim}(V_{t_1}^P,k_D)\subseteq \mathsf{Trim}(V_{t_2}^Q,k_D)$,
with probability $1-\negl(\lambda)$.}
\label{fig:dcp}
\end{figure}


\begin{remark}[Consistency of DAG properties]
Definitions~\ref{def:DG} (DAG Growth),~\ref{def:DQ} (DAG Quality), and~\ref{def:DCP} (DAG Common Past) are all stated relative to an honest party’s fork-choice tip and its preferred branch. This ensures consistency across the properties: \emph{DG} guarantees linear progress on some honest preferred chain, \emph{DQ} guarantees that honest blocks dominate the frontier of that chain, and \emph{DCP} guarantees that honest parties’ preferred branches agree on a long common past. Together, these properties provide a coherent DAG analogue of the chain-growth, chain-quality, and common-prefix properties from backbone analyses.
\end{remark}



%\newpage
\subsection{Ledger Security Properties}

\subsubsection{Persistence (Safety)}
A protocol has \emph{persistence} with parameter $k$ if for any two honest times $t_1\le t_2$,
the ledgers induced by the fork-choice at those times agree on all transactions that are at least $k$-deep at time $t_1$,
except with probability $\negl(\lambda)$.


\begin{figure}[htp!]
\centering
\begin{tikzpicture}[x=0.16cm,y=0.6cm,>=Latex,thick]
  % Axis
  \draw[-{Latex}] (0,0) -- (98,0) node[below right] {slots};

  % t1 and t2 markers
  \draw[dashed] (70,-2.0) -- (70,3.0) node[above] {$t_1$};
  \draw[dashed] (90,-2.0) -- (90,3.0) node[above] {$t_2$};

  % k-deep prefix at t1
  \fill[green!12] (0,1.4) rectangle (45,2.6);
  \draw[green!40!black] (0,1.4) rectangle (45,2.6);
  \node[green!40!black,anchor=west] at (0.5,2.0) {\scriptsize $k$-deep prefix at $t_1$};

  % Same prefix at t2 (visual echo)
  \draw[green!40!black,dashed] (0,1.4) rectangle (45,2.6);
  \node[green!40!black] at (22.5,1.0) {\scriptsize remains identical at $t_2$};

  % Legend arrow
  \draw[->] (46,2.0) -- (68,2.0) node[midway,above] {\scriptsize immutable region};
\end{tikzpicture}
\caption{\textbf{Persistence (Safety).} 
\small 
All but the last $k$ transactions in the ledger at time $t_1$ remain immutable and appear in every honest ledger at any later time $t_2\ge t_1$, except with probability $1-\negl(\lambda)$. 
The green region shows the $k$-deep finalized prefix at $t_1$, which remains unchanged in later ledgers.}

\label{fig:persistence}
\end{figure}

\begin{definition}[Persistence (Safety)]
Fix $k \in \mathbb{N}$. For an honest party $P$ and slot $t$, let $L_t^P$ denote
the ledger (totally ordered list of transactions) induced by applying the fork-choice
rule to $V_t^P$. We say $\mathrm{Persistence}[k]$ holds if for all pairs of honest
parties $P,Q$ and all slots $t_1 \le t_2$ after $\mathrm{GST}$,
\[
\Pr\!\Big[\, L_{t_1}^P[1..n-k] \;=\; L_{t_2}^Q[1..n-k] \,\Big] \;\ge\; 1-\negl(\lambda),
\]
where $n=|L_{t_1}^P|$. That is, all but the last $k$ transactions in $P$'s ledger
at time $t_1$ are immutable and appear in $Q$'s ledger at time $t_2$.
\end{definition}

\begin{remark}[On DAG Growth and Persistence]
The proof of persistence later in Sections~3.4.2 and~4.3.2 relies on DG 
to guarantee that honest blocks accumulate at a linear rate \emph{along honest preferred branches}. 
This ensures that once a block is $k$-deep under a fork-choice tip, the honest margin behind it 
grows steadily, preventing adversarial branches from overturning it. 
\end{remark}




\subsubsection{Liveness}



A protocol has \emph{liveness} with parameter $\ell_{\mathrm{live}}$ if for any valid transaction broadcast continuously by an honest party,
the transaction appears in the ledger of every honest party within $\ell_{\mathrm{live}}$ slots after $\mathrm{GST}$,
except with probability $\negl(\lambda)$.



\begin{figure}[htp!]
\centering
\begin{tikzpicture}[x=0.16cm,y=0.6cm,>=Latex,thick]
  % Axis
  \draw[-{Latex}] (0,0) -- (98,0) node[below right] {slots};

  % Broadcast start s and window [s, s+ell_live]
  \draw[very thick,|-|] (40,-0.8) -- (75,-0.8);
  \node[below] at (40,-0.8) {\scriptsize $s$};
  \node[below] at (75,-0.8) {\scriptsize $s+\ell_{\mathrm{live}}$};
  \node[below] at (57.5,-0.9) {\scriptsize inclusion window};

  % Broadcast event
  \fill[red!70] (40,-0.8) circle (1.3pt);
  \node[below=4pt,red!70] at (40,-0.8) {\scriptsize $\tx$ broadcast repeatedly};

  % Inclusion arrow into ledger
  \draw[->,red!70] (68,-0.3) -- (68,1.2);
  \node[red!70,align=center] at (68,1.5) {\scriptsize $\tx$ appears in $L_t^Q$\\[-1pt]\scriptsize for some $t\in[s,s+\ell_{\mathrm{live}}]$};
\end{tikzpicture}
\caption{\textbf{Liveness.} 
\small
Any valid transaction broadcast continuously by an honest party after GST is included in every honest ledger within $\ell_{\mathrm{live}}$ slots, except with probability $1-\negl(\lambda)$. 
Red marks the transaction broadcast event at slot $s$ and the inclusion window $[s,s+\ell_{\mathrm{live}}]$, with an arrow showing when the transaction enters an honest ledger.}

%\caption{\textbf{Liveness.} Any valid transaction broadcast continuously by an honest party after GST is included in every honest ledger within $\ell_{\mathrm{live}}$ slots, except with probability $1-\negl(\lambda)$.}
\label{fig:liveness}
\end{figure}


\begin{definition}[Liveness]
Fix $\ell_{\mathrm{live}} \in \mathbb{N}$. For any valid transaction $\tx$ that is
broadcast by an honest party $P$ in every slot after $\mathrm{GST}$, let
$E_t^Q(\tx)$ denote the event that honest party $Q$'s ledger $L_t^Q$ at the end of slot $t$
contains $\tx$. We say $\mathrm{Liveness}[\ell_{\mathrm{live}}]$ holds if for all honest parties $P,Q$
and for all slots $s \ge \mathrm{GST}$,
\[
\Pr\!\Big[\, \exists\, t \in [s,\,s+\ell_{\mathrm{live}}] : E_t^Q(\tx) \,\Big] \;\ge\; 1-\negl(\lambda).
\]
In words: any transaction repeatedly injected by an honest party after $\mathrm{GST}$
is guaranteed to appear in every honest party’s ledger within $\ell_{\mathrm{live}}$ slots,
except with negligible probability.
\end{definition}

\begin{remark}[On DAG Growth and Liveness]
Liveness requires that valid transactions eventually appear in all honest ledgers. 
The revised DG property, formulated with respect to preferred branches, 
ensures that honest block production makes linear progress in those branches. 
Coupled with TB (to prevent tips from fragmenting indefinitely) 
and DQ (to ensure honest dominance), this guarantees that transactions 
broadcast after GST will be absorbed and confirmed within bounded time. 
\end{remark}








\subsection{Tip Boundedness and Its Necessity}
\label{subsec:tip-boundedness}

The preceding properties (DAG Growth, DAG Quality, DAG Common Past) capture
progress, fairness, and eventual agreement on history. However, these three
alone are insufficient to guarantee a secure ledger. We introduce the additional
invariant of \emph{Tip Boundedness} (TB).

\begin{definition}[Tip Boundedness]
A protocol satisfies $\mathsf{TB}[\beta]$ if, at any time after GST, the number
of pairwise \emph{incomparable honest tips} (tips that are not ancestors of one
another) is at most $\beta$, except with negligible probability. Here $\beta$ is
$O(\Delta \cdot \text{block rate})$, i.e., proportional to the maximum number of
honest blocks that can be created during the delay window $\Delta$.
\end{definition}

\begin{remark}[Why this formulation]
Unlike DG, DQ, and DCP which can be interpreted in a linearized setting,
TB explicitly acknowledges the DAG’s concurrency: multiple tips may
exist at once. Bounding their number ensures that honest blocks regularly
\emph{merge or coalesce} tips, preventing indefinite fragmentation of honest
work. This makes DG, DQ, and DCP meaningful in the DAG setting, because they
now apply to a bounded set of preferred branches rather than an unbounded
frontier.
\end{remark}

\paragraph{Why TB Is Necessary.} \mnote{This is very informative but might not be worthy to fit it under 15-pages limit the whole thing. Or we could move it appendix?}
If the number of honest tips is unbounded, then:

\begin{itemize}
  \item \textbf{Safety problem:} Honest work fragments across many incomparable
  branches. In fork choice, an adversary can concentrate its weight on one branch
  to override the others. Even if DG, DQ, and DCP hold, safety fails because no
  branch accumulates a \emph{stable majority weight}. TB is necessary to limit
  the number of competing tips. By itself, TB does not guarantee that one branch
  immediately holds a majority (honest weight could still be split across a few
  tips), but it creates conditions under which honest cross-referencing and
  fork-choice dynamics ensure that one branch eventually dominates.
  \begin{remark}[Stable majority nuance]
  TB is thus a \emph{necessary but not sufficient} condition: it bounds
  fragmentation, and in combination with DG and DQ it yields eventual honest
  majority on one branch, assuming honest nodes cross-reference tips once visible.
  \end{remark}

  \item \textbf{Liveness problem:} Some honest blocks may remain stranded: if new
  tips appear faster than they can be merged, many honest transactions never
  receive enough descendants to be included in the common past. Liveness then
  fails, even though the DAG continues to grow.
  \begin{remark}[Transaction redundancy assumption]
  This liveness failure is most acute under the assumption that honest proposers
  do not re-include transactions once they have appeared in a block. If
  transactions from stranded blocks are eventually recycled into the mempool and
  included again, liveness of the \emph{transactions} can be salvaged, though at
  the cost of redundancy and wasted effort. We adopt the common assumption of
  no-transaction-redundancy to highlight why TB is critical.
  \end{remark}
  
% ============================= THIS PART IS NEW: Sept 11, 2025=============================
\paragraph{Natural redundancy, and verifiable random selection (VRS).}
The no-redundancy assumption highlights a sharp failure mode without TB, but real DAG deployments
typically exhibit \emph{natural redundancy}: multiple honest proposers include the same transaction across
concurrent branches. This redundancy cushions liveness: even if some blocks become stranded, the \emph{transaction}
still appears on other branches that the fork-choice later prefers. \mnote{This redundancy also reduces the effectiveness of a reorg execution. Attacking a single block might not be effective as if it introduces a conflict to a "redundant" transaction then the adversary would need to fight against other branches that include that transaction. So adversary cannot just simply attack a single block it must attack all blocks that its conflicting transaction conflicts with. Hence, higher redundancy yields more expensive reorg. -- This is one of the things I've mentioned during our last chat that I wanted to highlight.}

To make redundancy analyzable and tunable, we specify a simple \emph{verifiable random selection (VRS)}
policy for transaction sampling:
\[
\text{Include } tx \text{ in slot } s \text{ by validator } v \iff H(\rho_s \,\|\, \texttt{txid}(tx) \,\|\, pk_v) < q\cdot 2^\kappa ,
\]
where $\rho_s$ is a public, unpredictable per-slot seed (e.g., derived from the VRF beacons already present),
$q\in(0,1)$ controls expected inclusion probability per proposer, and $H$ is a hash modeled as a PRF.
This yields (i) publicly verifiable, non-manipulable inclusion decisions; (ii) independence across validators;
and (iii) a controllable \emph{redundancy factor}. If $m$ honest proposers are eligible in slot $s$, the number of
honest inclusions for any fixed $tx$ in that slot is $\mathrm{Binomial}(m,q)$.

\emph{Liveness under bounded redundancy.} Suppose each valid $tx$ is (re)broadcast after GST until included,
and VRS is used with parameter $q$ while the expected number of honest proposers per slot is $\lambda_h$.
Then in $O\!\big(\tfrac{1}{q\lambda_h}\big)$ slots, $tx$ is included in $\tilde\Theta(q\lambda_h)$ honest blocks across the last
$w$ slots (w.h.p.). Coupled with TB (which bounds the number of competing tips) and DG/DQ, this makes
it overwhelmingly likely that at least one of those copies lies on the eventual preferred branch, yielding
liveness even if some redundant copies are stranded.

\emph{Tuning redundancy vs.\ bandwidth.} The parameter $q$ sets the expected duplication rate:
higher $q$ improves time-to-inclusion but increases bandwidth/verification; smaller $q$ reduces duplication
yet still preserves liveness as long as $q\lambda_h$ remains bounded away from~$0$.
% ============================================================================

\end{itemize}




\begin{lemma}[Necessity of TB for Ledger Properties]
DAG Growth, DAG Quality, and DAG Common Past are insufficient to imply persistence
and liveness of the ledger. In particular, without Tip Boundedness there exist
adversarial schedules where: (i) some honest blocks never accumulate any confirming
descendants (breaking liveness, under the no-redundancy assumption), and (ii) two
branches each accumulate enough honest blocks to satisfy DG, DQ, DCP, yet neither
dominates, allowing divergent ledgers (breaking safety).
\end{lemma}

\begin{proof}[Proof sketch]
Consider a DAG where honest proposers perpetually extend different tips (e.g., due
to adversarially scheduled network delays). Both branches keep growing linearly
(DG holds), each sees an honest fraction (DQ holds), and occasional cross-references
ensure a large common prefix (DCP holds). However, without TB, tips are never fully
merged: two branches may each gain infinitely many honest blocks without one ever
outweighing the other. One can construct a schedule where two honest groups alternate
extending two different tips, resulting in ``stalemated'' superchains. Honest
transactions included in one branch never appear in the ledger of parties following
the other branch (violating liveness, assuming no transaction redundancy), and a
block confirmed in one branch can later be overturned by the other branch’s growth
(violating persistence).\mnote{To my understanding TB defines how many tips we can have concurrently but doesn't force to use all of them (force number of refs). So I don't see how this example with extending two different tips by two honest groups is addressed by TB. It's a split horizon case, but this would also require an adversary that is splitting the view of honest nodes. Otherwise, an honest node would need to include the other tip in his refs set and would merge both superchains. Right?}
\end{proof}

\paragraph{Remark: TB vs.\ DCP and Relation to Long-Delay Attacks.}
DAG Common Past (DCP) does \emph{not} imply Tip Boundedness (TB). DCP ensures that if one trims away the most recent $k_D$ layers, then all honest views coincide on the older prefix. However, DCP does not constrain how many incomparable tips may exist within those last $k_D$ layers. In particular, one can have a DAG where the deep past is fully common (DCP holds), yet the active frontier contains $\Theta(\Delta \cdot \text{block rate})$ or even unboundedly many tips. Thus, persistence and liveness may fail even though DG, DQ, and DCP all hold: the honest work is fragmented across too many branches to form a stable margin.

This parallels the \emph{long-delay attack} described by Pass, Seeman, and shelat in their analysis of Nakamoto consensus in asynchronous networks \cite[Section~8]{EC:PasSeeShe17}. When network delays are large relative to block production, the honest chain fragments into many concurrent forks, which an adversary can exploit to cause deep reorgs. Our TB property directly precludes such fragmentation in DAG protocols: it enforces a bound on the width of the tip set, thereby ensuring that honest references continually \emph{merge or coalesce} the frontier. In this sense, TB is the DAG analogue of the synchrony assumption needed to rule out long-delay attacks in chain-based protocols.


%%%%%%%%%%%%%%%%%%%%%%%%%%%%%%%%%%%%%%%%%%%%%%%%%%%%%%%%%%%%%%%%%%%%%%%%%%%%%%%


